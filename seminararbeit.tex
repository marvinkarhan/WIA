\documentclass[conference,compsoc,final,a4paper]{IEEEtran}
\usepackage[utf8]{inputenx}
\usepackage{float} % for Table float parameter H

%% Bitte legen Sie hier den Titel und den Autor der Arbeit fest
\newcommand{\autoren}[0]{Karhan, Marvin}
\newcommand{\dokumententitel}[0]{Was sind Dark Patterns? Welchen Einfluss haben sie auf die Nutzer?}

\input{preambel} % Weitere Einstellungen aus einer anderen Datei lesen

\begin{document}

% Titel des Dokuments
\title{\dokumententitel}

% Namen der Autoren
\author{
  \IEEEauthorblockN{\autoren}
  \IEEEauthorblockA{
    Hochschule Mannheim\\
    Fakultät für Informatik\\
    Paul-Wittsack-Str. 10,
    68163 Mannheim
  }
}

% Titel erzeugen
\maketitle
\thispagestyle{plain}
\pagestyle{plain}

% Eigentliches Dokument beginnt hier
% ----------------------------------------------------------------------------------------------------------

% Kurze Zusammenfassung des Dokuments
\begin{abstract}
Abstract
\end{abstract}

% Inhaltsverzeichnis erzeugen
{\small\tableofcontents}

% Abschnitte mit \section, Unterabschnitte mit \subsection und
% Unterunterabschnitte mit \subsubsection
\section{Einleitung}
"“Dark pattern” means a user interface designed or manipulated with the substantial effect of subverting or impairing user autonomy, decisionmaking, or choice" \autocite{OAL2018}

\section{Zusätzliche Angaben}
\subsection{Zentrale Begriffe}
\begin{itemize}
\item Dark Pattern
\item Anti-Pattern
\item evil design
\item black hat UX
\item dark ux
\item Marktpsychologie
\item CCPA (California Consumer Privacy Act)
\item GDPR (General Data Protection Regulation)
\item Persuasive design
\item Deceptive design
\item Human-centered computing
\item Human computer interaction (HCI)
\end{itemize}

\subsection{Zeitplan}
Generell ist vor jeder Textuellen Abgabe endsprechend dem Umfang der abgabe eine Rechtschreibprüfung eines Dritten angesetzt.
\begin{table}[H]
\begin{tabular*}{\linewidth}{ @{\extracolsep{\fill}}l  l}
    \toprule
\textbf{Zeitraum}                   & \textbf{Geplante Tätigkeit}       \\
    \midrule
16.04 - 23.04                       & Weitere Literaturrecherche        \\
24.04 - 30.04                       & Kapitel 3.0 (Kapitel einleitung), 3.1                          \\
01.05 - 14.05                       & Kapitel 3.2 - 3.4                          \\
\textbf{14.05}                      & \textbf{Abgabe des Probekapitels} \\
15.05 - 28.05                       & Kapitel 4                           \\
29.05 - 11.06                       & Kapitel 5                          \\
12.06 - 25.06                       & Kapitel Abstract und Einleitung                          \\
\textbf{25.06}                      & \textbf{Abgabe zum Peer-Review}   \\
\textbf{07.07}                      & \textbf{Peer-Review (8:00 - 11:15 Uhr)} \\
\textbf{08.07}                      & \textbf{Abgabe der Peer-Review-Bögen}   \\
09.07 - 16.07                       & Einarbeitung der Kritik und Letzte Verbesserungen                           \\
\textbf{16.07}                      & \textbf{Abgabe des Finalen Papers}      \\
    \bottomrule
\end{tabular*}
\end{table}

\subsection{Kontextabgrenzung}
Dieses Paper vermittelt einen Einblick in die Konzepte die das Fundament für die Definition des Dark Pattern Begriffes liefern. Wie sich Dark Patterns in Applikationen manifestieren und wie Verbraucher vor misbrauch durch Dark Patterns geschützt werden, bzw. wie können sich Nutzer davor schützen. Außerdem wird im Ausblick darauf eingegangen wie Regularien oder Nutzerverhalten den einsatz von Dark Pattern in zukunft beeinflussen kann.


\section{Entstehung des Dark Pattern Begriffes}
// David Brignull als erfinder des Begriffs 2010
\subsection{Anti-Pattern}
// Anti-Pattern als überbegriff von Dark-Pattern und einleitung in das thema
\subsection{Psychologie}
// Wie werden Nutzer von Dark Pattern beeinflusst?
// Diskussion verschiedener erhobenen Statistiken zu Nutzerverhalten
\subsection{Nudging}
// Vorhersehbares beeinflussen von Nutzern um ein gewünschtes ziel zu erreichen
\subsection{Wachstums Manipulation}
// Mittel zur verbreitung eines Produkts
// Amazon mit dem beispiel Reckless.com

\section{Dark Pattern Arten}
// Taxonomy mit verschiedenen Quellen belegen
\subsection{Nagging}
// Nerven bis man ja sagt (Z.B. IPhone apple pay einrichtung or no button for no)
\subsection{Behindernd}
// Roach motel, ez to enter hard to leave
\subsection{Irreführend}
// Verbergen von information durch design, Z.B. wenn ein shop immer direkt die teuerste variante (farblich) eines Produkts zeigt
\subsection{Heimlich}
// versteckte kosten, sneak into basked, versteckte subscriptins Z.B. hinter einem Free trial
\subsection{Zwingend}
// wenn man gezwungen wird etwas zu tun was man nicht will und was nicht erforderlich ist, Z.B. annehmen eines Newsletters zur anmeldung oder das liken einer seite um sie zu besuchen


\section{Verbraucherschutz}
\subsection{Gesetzliche Einschränkungen}
// Self-regulate or get regulated
\subsection{Nutzer Aufklärung}
// Nutzer aufklären, führt zur reduzierung des Erfolges von Dark Patterns, was wiederum dafür sorgt das Dark Pattern sich für firmen weniger rentieren und sie auf diese verzichten

\section{Ausblick}

\section{Fazit}

\nocite{*}
% Literaturverzeichnis
\addcontentsline{toc}{section}{Literatur}
\printbibliography
\end{document}
