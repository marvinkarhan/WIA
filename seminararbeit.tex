\documentclass[conference,compsoc,final,a4paper]{IEEEtran}
\usepackage[utf8]{inputenx}
\usepackage{float} % for Table float parameter H

%% Bitte legen Sie hier den Titel und den Autor der Arbeit fest
\newcommand{\autoren}[0]{Karhan, Marvin}
\newcommand{\dokumententitel}[0]{Welchen Einfluss haben Dark Patterns auf Nutzer?}

% Hie muss normalerweise nichts angepasst werden
\usepackage[pdftex]{graphicx}
\graphicspath{{img/}}
\DeclareGraphicsExtensions{.pdf,.jpeg,.jpg,.png}
\usepackage[cmex10]{amsmath}
\usepackage{algorithmic}
\usepackage{array}
\usepackage{dblfloatfix}
\usepackage{url}
\usepackage[autostyle=true,german=quotes]{csquotes}
\usepackage[backend=biber,
            sorting=none,   % Keine Sortierung
            doi=true,       % DOI anzeigen
            isbn=false,     % ISBN nicht anzeigen
            url=true,       % URLs anzeigen
            maxnames=6,     % Ab 6 Autoren et al. verwenden
            minnames=1,     % und nur den ersten Autor angeben
            style=ieee,]{biblatex}
\usepackage{booktabs}
\usepackage{xcolor}
\usepackage{listings}             % Source Code listings
\usepackage[printonlyused]{acronym}
\usepackage{fancyvrb}
\usepackage{tocloft} % Schönere Inhaltsverzeichnisse

% Farben definieren
\definecolor{linkblue}{RGB}{0, 0, 100}
\definecolor{linkblack}{RGB}{0, 0, 0}
\definecolor{darkgreen}{RGB}{14, 144, 102}
\definecolor{darkblue}{RGB}{0,0,168}
\definecolor{darkred}{RGB}{128,0,0}
\definecolor{comment}{RGB}{63, 127, 95}
\definecolor{javadoccomment}{RGB}{63, 95, 191}
\definecolor{keyword}{RGB}{108, 0, 67}
\definecolor{type}{RGB}{0, 0, 0}
\definecolor{method}{RGB}{0, 0, 0}
\definecolor{variable}{RGB}{0, 0, 0}
\definecolor{literal}{RGB}{31,0, 255}
\definecolor{operator}{RGB}{0, 0, 0}

\usepackage[ngerman]{betababel}

\DefineBibliographyStrings{ngerman}{
    andothers = {{et al\adddot}},  % Immer et al. sagen, auch bei Deutsch als Sprache
}
\usepackage[
      unicode=true,
      hypertexnames=false,
      colorlinks=true,
      colorlinks=false,
      linkcolor=darkblue,
      citecolor=darkblue,
      urlcolor=darkblue,
      pdftex
   ]{hyperref}
%	 \PrerenderUnicode{ü}


% Einstellungen für Quelltexte
\lstset{
    xleftmargin=0.1cm,
    basicstyle=\scriptsize\ttfamily,
    keywordstyle=\color{keyword},
    identifierstyle=\color{variable},
    commentstyle=\color{comment},
    stringstyle=\color{literal},
    tabsize=2,
    lineskip={2pt},
    columns=flexible,
    inputencoding=utf8,
    captionpos=b,
    breakautoindent=true,
    breakindent=2em,
    breaklines=true,
    prebreak=,
    postbreak=,
    numbers=none,
    numberstyle=\tiny,
    showspaces=false,      % Keine Leerzeichensymbole
    showtabs=false,        % Keine Tabsymbole
    showstringspaces=false,% Leerzeichen in Strings
    morecomment=[s][\color{javadoccomment}]{/**}{*/},
    literate={Ö}{{\"O}}1 {Ä}{{\"A}}1 {Ü}{{\"U}}1 {ß}{{\ss}}2 {ü}{{\"u}}1 {ä}{{\"a}}1 {ö}{{\"o}}1
}

\hypersetup{
    pdftitle={\dokumententitel},
    pdfauthor={\autoren},
    pdfdisplaydoctitle=true,
    hidelinks
}

% Makros für typographisch korrekte Abkürzungen
\newcommand{\zb}[0]{z.\,B.\ }
\newcommand{\dahe}[0]{d.\,h.\ }
\newcommand{\ua}[0]{u.\,a.\ }

% Wo liegt Sourcecode?
\newcommand{\srcloc}{src/}

% Literatur einbinden
\addbibresource{literatur.bib} % Weitere Einstellungen aus einer anderen Datei lesen

\begin{document}

% Titel des Dokuments
\title{\dokumententitel}

% Namen der Autoren
\author{
  \IEEEauthorblockN{\autoren}
  \IEEEauthorblockA{
    Hochschule Mannheim\\
    Fakultät für Informatik\\
    Paul-Wittsack-Str. 10,
    68163 Mannheim
  }
}

% Titel erzeugen
\maketitle
\thispagestyle{plain}
\pagestyle{plain}

% Eigentliches Dokument beginnt hier
% ----------------------------------------------------------------------------------------------------------

% Kurze Zusammenfassung des Dokuments
\begin{abstract}
Abstract
\end{abstract}

% Inhaltsverzeichnis erzeugen
{\small\tableofcontents}

% Abschnitte mit \section, Unterabschnitte mit \subsection und
% Unterunterabschnitte mit \subsubsection
\section{Einleitung}
"“Dark pattern” means a user interface designed or manipulated with the substantial effect of subverting or impairing user autonomy, decisionmaking, or choice"

\section{Zusätzliche Angaben}
\subsection{Zentrale Begriffe}
\begin{itemize}
\item Dark Pattern
\item Anti-Pattern
\item evil design
\item black hat UX
\item dark ux
\item Marktpsychologie
\item CCPA (California Consumer Privacy Act)
\item GDPR (General Data Protection Regulation)
\item Persuasive design
\item Deceptive design
\item Human-centered computing
\item Human computer interaction (HCI)
\end{itemize}

\subsection{Zeitplan}
Generell ist vor jeder textuellen Abgabe entsprechend dem Umfang der Abgabe eine Rechtschreibprüfung eines Dritten angesetzt.
\begin{table}[H]
\begin{tabular*}{\linewidth}{ @{\extracolsep{\fill}}l  l}
    \toprule
\textbf{Zeitraum}                   & \textbf{Geplante Tätigkeit}       \\
    \midrule
16.04 - 23.04                       & Weitere Literaturrecherche        \\
24.04 - 30.04                       & Kapitel 3.0 (Kapiteleinleitung), 3.1                          \\
01.05 - 14.05                       & Kapitel 3.2 - 3.4                          \\
\textbf{14.05}                      & \textbf{Abgabe des Probekapitels} \\
15.05 - 28.05                       & Kapitel 4                           \\
29.05 - 11.06                       & Kapitel 5                          \\
12.06 - 25.06                       & Kapitel Abstract und Einleitung                          \\
\textbf{25.06}                      & \textbf{Abgabe zum Peer-Review}   \\
\textbf{07.07}                      & \textbf{Peer-Review (8:00 - 11:15 Uhr)} \\
\textbf{08.07}                      & \textbf{Abgabe der Peer-Review-Bögen}   \\
09.07 - 16.07                       & Einarbeitung der Kritik und letzte Verbesserungen                           \\
\textbf{16.07}                      & \textbf{Abgabe des fertigen Papers}      \\
    \bottomrule
\end{tabular*}
\end{table}

\subsection{Kontextabgrenzung}
Dieses Paper vermittelt einen Einblick in die Konzepte, die das Fundament für die Definition des Dark Pattern Begriffes liefern. Wie sich Dark Patterns in Applikationen manifestieren und wie Verbraucher vor Missbrauch durch Dark Patterns geschützt werden, bzw. wie können sich Nutzer davor schützen. Außerdem wird im Ausblick darauf eingegangen wie Regularien oder Nutzerverhalten den Einsatz von Dark Pattern in Zukunft beeinflussen kann.

% TODO: remove after first turn in
\newpage
\section{Dark Patterns Grundlagen}
% // David Brignull als Erfinder des Begriffs 2010\\
% // wer setzt das warum ein\\
Designer machen nur was von ihnen gefordert wird, wenn sie es nicht tun werden es andere tun \autocite{Nerdwriter1_YT_2018}. Nutzer lesen nicht jedes Wort auf einer Webseite, sie Überfliegen und machen annahmen \autocite{Brignull}. Firmen können das ausnutzen indem sie die Seite anders aussehen lassen als, was sie Tatsächlich aussagt \autocite{Brignull}.
\subsection{Anti-Pattern}
% // Anti-Pattern als Überbegriff von Dark Pattern und Einleitung in das Thema\\
Ein Pattern ist der Bauplan einer Lösung zu einem wiederkehrenden Problem. Sie existieren in vielen Anwendungs Bereichen \autocite[S. 1-22]{MacDonald2019}. Anti-Pattern sind ein Sammel-Begriff für Pattern, welche wiederkehrende Lösungen liefern, aber dabei mehr Probleme erzeugen als Lösen \autocite[S. 193-222]{MacDonald2019}. Wie sich aus dem Namen bereits erschließen lässt sind Dark Patterns eine spezifische Pattern Art. Dark Patterns erzeugen in erster Linie Probleme für den Nutzer. Die Probleme die sie lösen wollen, liegen auf der Seite der jenegen die sie einsetzen. Weil hierbei der Nutzer ausgenutzt wird und Profit über Nutzerfreundlichkeit gestellt wird \autocite{Chivukula_2019}, zählen Dark Patterns zu der Familie der Anti-Pattern.

\subsection{Psychologie}
\label{chap:Psychologie}
% // Warum spielt Psychologie eine rolle bei Dark Patterns?
% \\// Wie werden Nutzer von Dark Pattern beeinflusst?
% \\// Diskussion verschiedener erhobenen Statistiken zu Nutzerverhalten:\\
Dark Patterns nutzen Menschliches verhalten aus, um Nutzer zu bewegen etwas ungewollt oder unbewusst zu tun \autocite{Brignull}. Dafür nutzen sie oft Kognitive Verzerrung aus. Kognitive Verzerrung beschreibt, wie mithilfe von effekten und techniken die Denkweise unseres Gehirns ausgenutzt werden kann \autocite{Mathur2019}.
Eine weit verbreitete Psychologische technik im Einzelhandel ist die psychologische Preisgestaltung. Das heißt der Preis eines Produkts wird minimal, unter einer Runden Zahl angesetzt. Diese technik ist schon seid mehreren Jahrzehnten im Einsatz und laut \citeauthor{Bizer_2005} (\citedate{Bizer_2005}) ein effektives mittel zur Verkaufssteigerung. Im gegensatz dazu steht \citeauthor{Wieseke_2015} Studie aus dem Jahr \citedate{Wieseke_2015}, die sagt: Runde Preise sorgen für die höchstmögliche Verkaufswahrscheinlichkeit, da diese Bequemer für den Käufer sind. Einen Vergleichbarer effekt kann bei Dark Patterns auftreten. Im ersten schritt sorgt der Einsatz von Dark Pattern für eine Höhere Nutzerbindung. Jedoch im zweiten Schritt zu einer gegenläufigen Wirkung \autocite*{M.Bhoot2020}.
\subsection{Nudging}
// Vorhersehbares Beeinflussen von Nutzern um ein gewünschtes Ziel zu erreichen

\subsection{Zweck von Dark Patterns}
// in die einleitung des themas
// Mittel zur Verbreitung eines Produkts
// Amazon mit dem Beispiel relentless.com

% TODO: remove after first turn in
\newpage
\section{Kategorien von Dark Patterns}
// Taxonomy mit verschiedenen Quellen belegen \autocite*{Gray_2018,M.Bhoot2020,Brignull}
\subsection{Nagging}
// Nerven bis man ja sagt (Z.B. iPhone Apple Pay-Einrichtung or No Button for No)
\subsection{Behindernd}
// Roach motel, easy to enter hard to leave
\subsection{Irreführend}
// Verbergen von Information durch Design, z.B. wenn ein Shop immer direkt die teuerste Variante (farblich) eines Produkts zeigt
\subsection{Heimlich}
// Versteckte Kosten, sneak into basked, versteckte Subscriptions, z.B. hinter einem free trial
\subsection{Zwingend}
// Wenn man gezwungen wird, etwas zu tun, was man nicht will und was nicht erforderlich ist, z.B. Annehmen eines Newsletters zur Anmeldung oder das Liken einer Seite, um sie zu besuchen


\section{Schutz vor Dark Patterns}
\subsection{Gesetzliche Einschränkungen}
// Self-regulate or get regulated
\subsection{Entwickler-Aufklärung}
// Entwickler entwickeln oft Dark Pattern ohne das ihnen das bewusst ist
\subsection{Nutzer-Aufklärung}
// subreddit '/r/assholedesign' \autocite{Chivukula_2019}\\
// Brignull klärt auf seiner seite auf \autocite{Brignull}
// Nutzer aufklären, führt zur Reduzierung des Erfolges von Dark Patterns, was wiederum dafür sorgt, dass Dark Pattern sich für Firmen weniger lohnen und sie auf diese verzichten\\
Laut \citeauthor{Brignull} ist der beste Schutz vor Dark Patterns ist sie sich bewusst zu machen und die Firmen die sie benutzen zu boykottieren. \autoref{chap:Psychologie} zeigt das der Mensch anfällig für Dark Patterns ist \citeauthor{M.Bhoot2020}. fanden in einer Nutzerbefragung herraus, dass Nutzer Dark Patterns mit stark schwankender Konsistenz erkennen. Nur 18,6\% der Befragten haben das Roach Motel Dark Pattern erkannt \autocite{M.Bhoot2020}. Deshalb ist es wichtig Nutzer über die Dark Patterns aufzuklären, sodass sie diese leichter erkennen uns sich selbst vor ungewollten Konsequenzen schützen zu können.

\section{Fazit}
// enthält ausblick

\nocite{*}
% Literaturverzeichnis
\addcontentsline{toc}{section}{Literatur}
\printbibliography
\end{document}
// bill zitieren mit namen